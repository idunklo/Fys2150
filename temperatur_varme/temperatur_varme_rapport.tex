\documentclass[a4paper,norsk,12pt,oneside]{article}
%\documentclass[11pt,twocolumn]{article}

% To use norwegian
\usepackage[utf8]{inputenc}
\usepackage[T1]{fontenc}
\usepackage[norsk]{babel}
% math
\usepackage{amsmath}
\usepackage{amsfonts}
\usepackage{amssymb}

\usepackage{graphicx} % images
\usepackage{float}
\usepackage{enumerate}
\usepackage{fancyvrb} % code
\usepackage{algorithm2e} % algorithm
% For listing
\usepackage{listings} % code with color
\usepackage{courier}
\usepackage{caption}
\usepackage{color}

\title{Fys-2150\\ varme og temperatur}
\author{Idun Kløvstad}
\date{\today}  

% Command for L'Hopital's rule (requires extarrows-package)
\newcommand{\Heq}[1]{\xlongequal[\mathrm{L'H}]{\left[#1\right]}}

% Double underline
\newcommand{\dul}[1]{\underline{\underline{#1}}}

% Custom operators
\DeclareMathOperator{\nul}{Nul\,}
\DeclareMathOperator{\Proj}{Proj\,}
\DeclareMathOperator{\Sp}{Sp\,}
\DeclareMathOperator{\res}{Res\,}
\DeclareMathOperator{\Log}{Log\,}

% Allow displayed page breaks
\allowdisplaybreaks

% Settings for listings
\definecolor{dkgreen}{rgb}{0,0.6,0}
\definecolor{gray}{rgb}{0.5,0.5,0.5}
\definecolor{mauve}{rgb}{0.58,0,0.82}

\lstset{%
  %language=python,                     % the language of the code
  basicstyle=\footnotesize\ttfamily,   % the size of the fonts that are used for the code
  numbers=left,                        % where to put the line-numbers
  numberstyle=\tiny,                   % the style that is used for the line-numbers
  stepnumber=2,                        % the step between two line-numbers. If it's 1, each line will be numbered
  numbersep=5pt,                       % how far the line-numbers are from the code
  extendedchars=true,
  %backgroundcolor=\color{white},       % choose the background color. You must add \usepackage{color}
  %showspaces=false,                    % show spaces adding particular underscores
  showstringspaces=false,              % underline spaces within strings
  %showtabs=false,                      % show tabs within strings adding particular underscores
  %frame=single,                        % adds a frame around the code
  frame=b,                             % adds a line at the bottom
  %rulecolor=\color{black},             % if not set, the frame-color may be changed on line-breaks within not-black text (e.g. commens (green here))
  tabsize=2,                           % sets default tabsize to 2 spaces
  %captionpos=b,                        % sets the caption-position to bottom
  breaklines=true,                     % sets automatic line breaking
  breakatwhitespace=false,             % sets if automatic breaks should only happen at whitespace
  %title=\lstname,                      % show the filename of files included with \lstinputlisting; also try caption instead of title
  keywordstyle=\color{blue},           % keyword style
  commentstyle=\color{dkgreen}\textit, % comment style
  stringstyle=\color{mauve},           % string literal style
  %escapeinside={\%*}{*)},              % if you want to add LaTeX within your code
  %morekeywords={*,...},                % if you want to add more keywords to the set
  xleftmargin=-20pt,
  xrightmargin=-20pt,
  framexleftmargin=19pt,
  framexbottommargin=4pt,
  framexrightmargin=21pt,
} 

% caption for listing
\newlength{\mycapwidth}\setlength{\mycapwidth}{\textwidth}
\addtolength{\mycapwidth}{75pt}
\DeclareCaptionFont{white}{\color{white}}
\DeclareCaptionFormat{listing}{\colorbox[cmyk]{0, 0, 0, 0.8}
    {\parbox{\mycapwidth}{\hspace{15pt}#1#2#3}}}
\captionsetup[lstlisting]{format=listing,labelfont=white,textfont=white, 
    singlelinecheck=false, margin=-40pt, font={bf,footnotesize}}


\begin{document}
\maketitle
\newpage

\begin{abstract}
    Denne rapporten vil ta for seg temometri, samt måling av termisk diffusivitet. 
\end{abstract}

\section{Introduksjon og teori}

Det aktive elementet i termistorer som brukes til temperaturlåling er små biter av 
halvledere laget av f.eks. keramiske metalloksyder eller silisium og germanium. 
Termistoren funker slik at antallet elektroner som eksiterer til ledningsbåndet øker med 
temperaturen. Dermed går den elektriske motstanden, \(R\) raskt ned med økende temperatur. 
(hentet fra oppgaveteksten) 
Temperaturen til et objekt kan derfor bestemmes ved hjelp av en termistor og likningen 
under, likning \ref{eq:hart}, som kalles Steinhart-Hart-likningen. 

\begin{equation}\label{eq:hart}
    \frac{1}{T} = a + bln(R) + cln(R)^3
\end{equation} 

T er her temperaturen i Kelvin, R er den elektriske motstanden, målt i \(\Omega\) og a, b og c er tilpasningsparametre for kalibreringsmodellen. 

\section{Eksperimentelt}

\subsection{Termoelement og termistor}\label{subsec:termterm}

For å undersøke om et termoelement oppfører seg som foreventet, ble termospenningen over 
en isbit ble målt ved hjelp av et termoelement og et multimeter av typen, Fluke 45. Dette 
ble gjort ved å stikke endene av termoelenentet ned i hvert sitt 
lille hull i isbiten, for så å gjøre en måling. Resultatet blir fremstilt i tabell 
\ref{tab:isbit} under seksjon 
\ref{subsec:termres} 

Den ene enden av termoelementet ble så fjernet fra isbiten og festet, med tape, til en 
alluminiumskube som sto montert på bordet. Den andre enden av termoelementet forble i 
isbiten. Deretter ble termospenningen målt på nytt. Oppsettet fikk stå slik over en liten 
periode, slik at det kunne observeres om termospenningen endret seg med tid.  

I neste del av forsøket festes to termistorer til alluminiumskuben, i tilegg til 
termoelementet. De to termistorene ble festet på hver sin side av kuben. Motstandene over 
hver termistor ble målt med multimeter av typen Fluke 75. For å få gode målinger, må 
målingene stabelisere seg før et resultat kan skrives ned. Under dette forsøket måtte 
det ene multimeteret byttes ut før målingene sluttet å hoppe. Det tok mellom 30 og 
60 sekunder før målingene stabiliserte seg, etter at multimeteret ble skiftet ut. 
Målingene fra dette forsøket er å finne i tabell \ref{termterm} under seksjon
\ref{termres}. 

\subsection{IR-termometer}\label{subsec:ireksperiment}

Starten av dette forsøket består av å utforske hvordan et IR-termometer, av type Fluke 62, 
fungerer. Det første som ble gjort var derfor å måle forskjellige ting i rommet. Avstanden 
til det som ble målt på, ble bestemt ved å bruke meterstokk. Alle disse målingene ble 
gjort med liten vinkel til flatenormalen. Resultatene av målingene vises i tabell 
\ref{tab:ir1} under seksjon \ref{subsec:ir}.

Deretter ble det gjort målinger på samme alluminiumskube som ble brukt til beregninger 
beskrevet i seksjon \ref{subsec:termterm}. Målingene ble gjort med forskjellige vinkler, 
avstader og med forskjellig reflektert bakgrunn. Avstanden ble målt med meterstokk.
Tabell \ref{tab:ir2} viser resultatet av målingene. 

Videre i forsøket skulle IR-termometeret brukes på hver av sidene av Leslies kube. Leslies 
kube har også en innebygd termistor. På denne måten kunne temperaturen regnes ut på to 
forskjellige måter og resultatene kunne sammenliknes. Avstanden til Leslies kube ble målt med meterstokk.

\section{Resultater}

\subsection{Termoelement og termistor}\label{subsec:termres}

\begin{table}[H]  
  \begin{center}
      \caption{Tabellen viser resultatet av termospenning målt over en isbit, som 
          beskrevet i seksjon \ref{subsec:termterm} \label{tab:isbit}}
  \begin{tabular}{|c|} \hline
      \textbf{\(\epsilon_T\)} \\ \hline
  \(0.01 mV\)\\ \hline
  \end{tabular}
  \end{center}
\end{table}  

I tabellen under finnes målingene av motstand, \(R_1\) og \(R_2\), gjennom to termistorer 
på aluminiumskloss som beskrevet i seksjon \ref{subsec:termterm}, etter at målingene ble 
stabile, samt termospenningen, \(\epsilon_T\), som ble målt mellom kloss og isbit. Første 
måling i tabellen inkluderer ikke måling av motstand, fordi den ble gjort før termistorene 
ble koblet til kuben. Det ble observert at termospenningen mellom kuben og isen, ikke 
forandret seg nevneverdig over tid. Den ble muligens litt lavere. 

Med motstandene målt med termistorer, kan likning \ref{eq:hart} brukes til å beregne 
temperaturen. For å kunne relatere til informasjonen, ble det i dette tilfellet valgt å 
regne om fra Kelvin til Celsius. Motstanden som ble brukt i utregningene er et  
gjennomsnitt mellom det som ble målt av de to termistorene. 

\begin{table}[H]  
  \begin{center}
      \caption{Denne tabellen viser termospenning målt mellom en aluminiumskuben og en 
          isbit. Den viser også motstandene i to termistorer, festet til aluminiumskuben. 
          Dette er 
          beskrevet i seksjon \ref{subsec:termterm}. Siste kolonne viser også temperaturen 
          av kuben, beregnet ut ifra den elektriske motstanden \label{tab:kloss}}
  \begin{tabular}{|c|c|c|c|c|} \hline
      \textbf{måling \#} & \textbf{\(\epsilon_T [mV]\)} & \textbf{\(R_1 [k\Omega]\)} & 
      \textbf{\(R_2 [k\Omega]\)} & \textbf{\(T(R) [^\circ C]\)} \\ \hline
  \(1\) & \( 0.842 \) & & & \\ \hline
  \(2\) & \( 0.834 \) & \(114.8 \pm 0.2\) & \(114.1 \pm 0.1\) & \(22.12\) \\ \hline
  \(3\) & \( 0.837 \) & \(114.5 \pm 0.2\) & \(113.7 \pm 0.2\) & \(22.20\)\\ \hline 
  \end{tabular}
  \end{center}
\end{table} 

Den beregnede usikkerheten er i dette tilfelle mye mindre enn standardavviket. 

\subsection{IR-termometer}\label{subsec:ir}

Seksjon \ref{subsec:ireksperiment} beskriver målinger gjort med IR- termometeret Fluke 62.
Tabellene under, tabell \ref{tab:ir1} og \ref{tab:ir2} viser henholdsvis resultater fra 
målinger gjort i rommet og målinger gjort på alluminiumskloss. Tabell .. inneholder både 
resultater fra målinger gjort med IR-termometer på Leslies kube og målinger med kubens egne
innebygde termistor. 

\begin{table}[H]  
  \begin{center}
      \caption{Denne tabellen viser egendefinerte målinger gjort i rommet, for å bli kjent 
          med IR-termometeret, Fluke 62. \label{tab:ir1}}
  \begin{tabular}{|c|c|c|c|} \hline
      \textbf{Hva måles på?} & \textbf{målt temperatur \( [^\circ C]\)} & \textbf{avstand 
      [m]} \\ \hline
  Bord & \(23.4\) & \(0.4\)\\ \hline
  Vegg & \(20.2\) & \(15\)\\ \hline
  Stolpe & \(21.8\) & \(5\)\\ \hline
  \end{tabular}
  \end{center}
\end{table} 

\begin{table}[H]  
  \begin{center}
      \caption{I denne tabellen vises resultatene etter målinger på alluminiumskube med IR-  
          termometeret, Fluke 62. \label{tab:ir2}}
  \begin{tabular}{|c|c|c|c|c|} \hline
      \textbf{Avstand til kuben [m]} & \textbf{vinkel\( [^\circ]\)} & 
      \textbf{reflektert bakgrunn} & målt temperatur [\(^\circ C\)] \\ \hline
    \(0.15\) & \(\sim0\) & eflekterer på seg selv & \(24.4\)\\ \hline
  \(0.50\) & \(\sim0\) & reflekterer på seg selv & \( 23.8\)\\ \hline
  \(1.5\) & \(\sim0\) & reflekterer på seg selv & \(24.6\)\\ \hline
  \(2.0\) & \(\sim0\) & reflekterer på seg selv & \(23.4\)\\ \hline
  \(0.2\) & \(\sim 10\) & vegg & \(22.2\)\\ \hline
  \(0.35\) & \(\sim45\) & svart labjournal (matt) & \(24.0\)\\ \hline  
  %\(0.35\) & \(\sim45\) & hvitt ark & \(24.2\)\\ \hline
  \(0.35\) & \(\sim 45\) & Ole Gunnar & \(28.0\)\\ \hline
  \(1.0\) & \(\sim 45\) & svart labjournal (matt) & \(23.6\)\\ \hline
  \(1.0\) & \(\sim 45\) & hvitt ark & \(23.2\)\\ \hline
  \(1.0\) & \(\sim 45\) & Ole Gunnar & \(27.2\)\\ \hline
  \end{tabular}
  \end{center}
\end{table}

Avstanden fra IR-termometeret til Leslies kube da målingene ble gjort, ble målt til \(0.3m\).

\begin{table}[H]  
  \begin{center}
      \caption{Her presenteres målingene gjort med IR-termometer på Leslies kube, samt måling 
          gjort med kubens innebygde termistor \label{tab:ir1}}
  \begin{tabular}{|c|c|c|c|c|} \hline
      \textbf{side1(blank)} & \textbf{side2(matt hvit)} & \textbf{side3(matt svart)} & 
      \textbf{side4(metallisk)} & \textbf{motstand} \\ \hline
  \(25.0\) & \(51.2\) & \(51.2\) & \(30.4\) & \(31.447\)\\ \hline
  \end{tabular}
  \end{center}
\end{table}       


\section{Diskusjon}

\section{Konklusjon}

\bibliography{referanser}
\end{document}                                             
            
